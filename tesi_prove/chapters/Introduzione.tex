Quando bisogna recarsi presso una via mai sentita prima o verso un luogo che si frequenta saltuariamente, ci si affida spesso a servizi di navigazione per ottenere informazioni sul percorso più veloce da intraprendere.
Il mezzo di trasporto invece, nella maggior parte dei casi, viene scelto a priori dall'utente per diversi motivi quali abitudine, differenza di prezzo o mancanza di interesse per le alternative.

Lo stato attuale della tecnologia permette ai provider di questi servizi di avere una stima del tempo di percorrenza molto precisa e realistica, grazie a dati in tempo reale su traffico, incidenti, deviazioni di percorso e grazie allo storico delle tratte percorse dalle loro migliaia di utenti.

In questa tesi vengono usati tali valori sulla stima di percorrenza per confrontare ogni mezzo su un percorso comune, ripetendo tale confronto più volte lungo l'arco della giornata e variando il percorso richiesto, al fine di studiare le performance medie per ogni mezzo e di analizzarne eventuali variazioni per effettuare dei confronti.

Sebbene sia ragionevole pensare che l'auto di proprietà sia più veloce dei mezzi pubblici e delle biciclette, questo ragionamento potrebbe non reggere all'interno di una città come Milano. Infatti, secondo l'ultimo studio annuale di TomTomIndex del 2019\cite{tomtomindexmilan}, Milano rientra tra le prime 100 città su scala globale per livello di congestione stradale, con un traffico tale da far quasi raddoppiare i tempi di percorrenza in auto durante le ore di punta del mattino e del tardo pomeriggio. In aggiunta al fatto che alcuni mezzi non soffrono minimamente di questo problema, come le metropolitane, o ne soffrono parzialmente, come le biciclette, questo rende l'ipotesi di partenza più discutibile.

...studiare altri papers...