\section{Progettazione dello studio}

Per poter confrontare 

\section{Mezzi di trasporto e relativi servizi}

Sono stati presi in considerazione i principali mezzi di trasporto con cui è possibile spostarsi all'interno del Comune di Milano, e per ognuno di essi è stato selezionato un servizio di navigazione in grado di offrire stime del tempo di percorrenza dinamiche per tale mezzo. Nello specifico:

\begin{itemize}
	\item mezzi pubblici ATM e passanti ferroviari;
	\begin{itemize}
		\item HERE WeGo REST API\cite{herewegoapi}
	\end{itemize}
	\item automobile di proprietà;
	\begin{itemize}
		\item Waze API\cite{wazeapi}
	\end{itemize}
	\item bicicletta di proprietà;
	\begin{itemize}
		\item OpenStreetMap\cite{openstreetmap}
	\end{itemize}
	\item a piedi;
	\begin{itemize}
		\item OpenStreetMap\cite{openstreetmap}
	\end{itemize}
	\item car sharing Enjoy;
	\begin{itemize}
		\item Enjoy Map\cite{enjoycarsharing}
	\end{itemize}
\end{itemize}

\section{Scelta tra percorsi random e prefissati}

Si è scelto di usare diverse tratte su cui basare il confronto allo scopo di ricoprire a livello topografico una buona parte del Comune di Milano e di diversificare i confronti in base alle caratteristiche delle tratte quali lunghezza, coordinate geografiche di partenza e destinazione, per simulare al meglio la posizione di un possibile utente nella mappa.

Per rendere automatiche le richieste da inoltrare ai servizi di navigazione è stata creata una sorgente dal quale pescare le tratte. Sono stati presi in considerazione tre principali approcci per creare tale sorgente: hardcoded random; hardcoded di tratte realmente percorse; generazione a random just-in-time. I primi due approcci sono risultati fin da subito problematici.

In primo luogo, la scelta della destinazione avrebbe introdotto bias cognitivi riguardo il possibile utente del percorso, per esempio, selezionando tratte con delle università come punto di arrivo avrebbe portato ad analizzare la mobilità solamente dal punto di vista di una determinata utenza, nel caso dell'esempio, la categoria di studenti e dipendenti presso quelle determinate strutture. Altri problemi simili sarebbero sorti nello scegliere la partenza, la lunghezza, la distanza dal centro città e numerosità delle tratte.

Seconda problematica molto più rilevante è che, involontariamente, si sarebbero introdotti dei percorsi che avrebbero favorito un mezzo piuttosto che un altro. Nella città di Milano infatti si contanto numerosi tratti stradali di questo genere: strade con corsia preferenziale per mezzi pubblici e taxi; strade a singola corsia e con numerosi semafori (quindi più soggetta a incolonnamenti); tangenziali; tratti coperti da passanti ferroviari (più veloci delle metropolitane e con meno fermate). Tale scelta avrebbe portato ad analizzare dati non rappresentativi della città, con conseguenti risultati in un certo senso "falsati".

Usando l'approccio della generazione a random non solo si sarebbero evitati tali problemi, ma i problemi stessi si sarebbero trasformati in analisi da poter eseguire a posteriori, per esempio selezionando da tutte le tratte generate a random quelle che hanno portato nei pressi di un'università, o tratte che in linea d'aria hanno coperto particolari strade favorevoli a determinati mezzi di trasporto. Visti i vantaggi e la flessibilità dell'approccio, si è optato per quest'ultimo.

\section{Generatore random dei percorsi}

Prima ancora di scrivere una funzione per generare tratte a random è stata scelta l'area geografica all'interno della quale generarle. Siccome il servizio di car sharing Enjoy non permette di usare la propria flotta al di fuori del Comune di Milano, tale area è stata selezionata come terreno per i confronti, rappresentata sotto forma di rettangolo per questioni di semplicità nella programmazione. Nello specifico, sono state scelte le coordinate (45.450562\textdegree, 9.158959\textdegree) e (45.482032\textdegree, 9.206763\textdegree) rispettivamente come vertice in basso a sinistra e in alto a destra del rettangolo rappresentativo dell'area selezionata. In questo modo la generazione di punti a random è stata ridotta alla generazione di coordinate maggiori o uguali della prima e minori o uguali della seconda.

Una volta scritta la funzione per la generazione a random delle tratte, è stato introdotto un constraint appositamente per simulare dei percorsi scomodi da fare a piedi, ovvero per generare tratte che supererebbero i 20 minuti di camminata per essere coperte, e al tempo stesso che giustificherebbero l'utilizzo della macchina. Si è scelto di usare 2 km in linea d'aria dal punto di partenza a quello di destinazione come misura minima della lunghezza di una tratta.

Un secondo constraint è stato introdotto per introdurre equamente 4 tipologie diverse di viaggio...TODO...

...DA QUI IN POI E' ROBA VECCHIA...

\subsection{Acquisizione dei dati}

I dati alla base di questo studio sono stati forniti tramite A.P.I dai seguenti servizi: HereWeGo (here.com) per le soluzioni di percorrenza coi mezzi pubblici ATM; Waze (waze.com) per la percorrenza in macchina; OpenStreetMap (openstreetmap.org) per la percorrenza in bici e a piedi; ... per il car sharing.

\subsubsection{Soluzioni di percorrenza}

HereWeGo e Waze forniscono soluzioni di percorrenza in tempo reale, tenendo conto delle condizioni attuali del traffico e della posizione dei mezzi. In particolare, Waze prende in considerazione il traffico, gli incidenti, le strade chiuse per lavori e le deviazioni, mentre HereWeGo prende in considerazione la posizione dei mezzi pubblici in tempo reale, indicando tempi di attesa, rallentamenti e deviazioni di percorso. OpenStreetMap invece fornisce soluzioni di percorrenza statiche, sia pedovia che in bicicletta, che non dipendono quindi dal traffico o da altre condizioni stradali. I dati di quest'ultimo infatti sono stati usati solamente come termini di riferimento. Le A.P.I di tali servizi sono nella forma:
\begin{lstlisting}[language=Go]
Service.GetRoute(A, B coordinate) Solution
\end{lstlisting}
dove il risultato restituito, in formato JSON, contiene la miglior soluzione di percorrenza per il servizio richiesto. Tra i dati contenuti nel risultato sono presenti informazioni come durata stimata del tragitto, step di navigazione, avvisi e alternative.

\subsubsection{Car sharing}

Dato che non sono stati trovati servizi per stimare la percorrenza di una tratta col car sharing, che includesse la ricerca di una macchina vicina al punto di partenza e infine una stima del percorso, ne è stato programmato uno ad hoc. Il servizio ... offre tramite un A.P.I una lista aggiornata in tempo reale di tutte le macchine parcheggiate, libere di essere prenotate:
\begin{lstlisting}[language=Go]
Service.GetFreeCars() []CarInfo
\end{lstlisting}
dove per ogni entry nella lista sono presenti i dati della macchina e le coordinate geografiche della posizione.

Per creare un servizio in grado di fornire una soluzione di percorrenza in car sharing basata su dati in tempo reale sono stati uniti tre servizi, quello di OpenStreetMap per la percorrenza a piedi, quello di Waze per la percorrenza in macchina e quello di ... per avere la lista aggiornata delle macchine libere. Sono stati usati nel seguente modo:
\begin{lstlisting}[language=Go]
func GetCarSharingRoute(A, B) Solution {
	freeCars = CarSharingService.GetFreeCars()
	C = findClosestCar(freeCars, A)
	
	footRoute = OpenStreetMap.GetRoute(A, C)
	carRoute = Waze.GetRoute(C, B)
	
	return Solution{
		footRoute.EstimatedTravelTime +
			carRoute.EstimatedTravelTime,
		...
	}
}
\end{lstlisting}
Dato un utente "U" che vuole andare da un punto "A" a un punto "B" in car sharing, il servizio creato appositamente procede nel scaricare la lista aggiornata delle macchine libere, cerca tra queste la macchina più vicina al punto di partenza desiderato "A" e ne calcola il tragitto a piedi per raggiungerla (punto "C"), dopodichè calcola il resto del tragitto, dalla posizione della macchina "C" al punto di destinazione desiderato "B", in macchina. Con questo metodo si è ottenuta una stima in tempo reale della percorrenza in car sharing per ogni tratta, che ha permesso a posteriori di ricavare altre informazioni utili, come per esempio il tempo medio per raggiungere un auto.

Per questione di semplicità, la ricerca della macchina più vicina è stata implementata con una algoritmo greedy che cerca la macchina fisicamente più vicina al punto di partenza, ignorando totalmente il verso di marcia del tragitto, possibilmente aumentando il tempo di percorrenza in macchina.

\subsection{Attuazione delle richieste e salvataggio}

La simulazione è stata programmata nel seguente modo:
\begin{enumerate}
\item ogni 60 secondi viene generata una tratta a random col generatore precedentemente discusso
\item parallelamente viene richiesto a tutti i servizi di fornire la soluzione migliore per percorrere tale tratta
\item il risultato viene salvato sotto forma di file JSON e compresso. Il processo si ripete
\end{enumerate}
E' stato scelto un intervallo di 60 secondi per permettere a tutti i servizi di risolvere la richiesta tramite A.P.I e per non eccedere il limite massimo di richieste mensili offerte gratuitamente dal servizio di HereWeGo.



\section{Raccolta dati}

La simulazione è stata programmata per andare in esecuzione dalle ore 7:00 alle ore 23:59 di ogni giorno per avere un confronto più alla pari. Infatti, prima delle ore 7:00 i mezzi pubblici operano con orario ridotto e meno mezzi, vista la bassa richiesta in quella fascia oraria.

I dati riguardanti la simulazione sono stati raccolti a partire dall'1 marzo 2020 per un periodo di circa 4 mesi, fino al 10 luglio 2020. Vista l'emergenza per epidemia da coronavirus (giusto scritto così?), tutta l'Italia è entrata in uno stato di lockdown annunciato per decreto legge dal 12 marzo 2020, allentando gradualmente le restrizioni a partire dal 4 maggio 2020. I dati all'interno di questo intervallo sono quindi da considerarsi poco utili all'analisi.

\todo{atrent: mi dicevi che hai stoppato gli script, però io andrei avanti almeno fino a fine anno}