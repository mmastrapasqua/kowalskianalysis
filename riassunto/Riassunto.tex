\documentclass[a4paper,12pt]{article}
\usepackage[utf8]{inputenc}
\usepackage{hyperref}
\usepackage{refcheck}
\usepackage{cleveref}
\usepackage{hyperref}
\usepackage[italian]{babel}

\usepackage[a4paper,top=2cm,bottom=3cm,outer=2cm,verbose,headheight=1cm,heightrounded]{geometry}
%\setlength{\marginparwidth}{4.5cm} %per farci stare todonotes

\title{{\Large\textbf{CONFRONTO TRA AUTOMOBILE E \linebreak MEZZI DI TRASPORTO ALTERNATIVI \linebreak NEL COMUNE DI MILANO BASATO SU \linebreak STIME DI TEMPI DI PERCORRENZA}}}
\author{\LARGE{Mauro Mastrapasqua - 892629}}
\date{\Large{27/01/2021}}


\begin{document}
%	\pagenumbering{gobble}
	\maketitle
	\vspace{1cm}
	
\section{Obiettivo del lavoro}
Confrontare mezzi pubblici e privati sulla base di stime di percorrenza all'interno del Comune di Milano.

\section{Descrizione lavoro svolto}
L'idea iniziale era quella di usare dati (di una tesi in corso) di campionamento delle stazioni del bike sharing BikeMi e quelli delle auto libere dei servizi di car sharing Car2Go, Enjoy e Sharengo per estrarre da essi i tragitti effettuati dagli utenti, calcolarne il tempo impiegato a percorrerli e usare servizi di navigazione per calcolare stime di percorrenza degli stessi tragitti ma con mezzi di trasporto alternativi al fine di confrontarli e di analizzarne le performance singolarmente, con un particolare riguardo all'automobile di proprietà.
Per evitare di analizzare la mobilità dal punto di vista di una sola utenza e per problemi legati alle condizioni di traffico stradale, che variano per giorno e orario, si è deciso di non usare questi dati storici.

La scelta è invece caduta sui servizi di navigazione in grado di offrire stime di percorrenza in tempo reale per \textbf{tutti} i mezzi di trasporto presi in considerazione, e al posto di usare tragitti realmente percorsi si è deciso di crearli con una funzione random. In assenza di API, per alcuni servizi si è effettuato lo \textit{scraping} dall'applicazione web. Per il confronto sono stati scelti i seguenti mezzi di trasporto: automobile privata; car sharing Enjoy; mezzi pubblici ATM; bicicletta di proprietà; percorso a piedi. La raccolta dati è avvenuta per mezzo dei seguenti servizi: scraping di Waze, OpenStreetMap ed EnjoyMap; API di Here.

\`E stato progettato e realizzato un programma per creare tragitti random a ritmo di 1 al minuto e per chiedere di risolvere tali tragitti con ogni mezzo di percorrenza scelto, tramite il servizio o \textit{scraper} associato. La raccolta dati è iniziata il 1 marzo 2020 ed è terminata il 27 giugno 2020. Per questioni di coerenza sono stati presi in considerazione solamente i dati a partire dal 4 maggio 2020, data che coincide con l'inizio dell'allentamento delle misure di restrizione dovute all'emergenza COVID-19 che hanno completamente azzerato il traffico durante il lockdown. I risultati sono stati raccolti in file JSON e successivamente tradotti in un unico file CSV. Tramite l'utilizzo di notebook Jupyter sono stati elaborati i dati raccolti, prima analizzati nel loro insieme, successivamente puliti dagli errori e infine utilizzati per diversi calcoli. Si citano la variazione della velocità media di ora in ora per ogni mezzo, la differenza di velocità tra il pre e il post lockdown, la differenza di velocità in base alla lunghezza della tratta e infine il confronto vero e proprio tra auto privata e mezzi alternativi.

\section{Tecnologie coinvolte}
Linguaggio di programmazione Go e librerie di networking; console da sviluppatore di Firefox; notebook Jupyter; Bash scripting.

\section{Competenze e risultati raggiunti}

\subsection{Quali risultati sono stati raggiunti rispetto agli obiettivi iniziali?}
Sono stati raggiunti tutti gli obiettivi iniziali. In particolare, sono state completate le analisi e le comparazioni di tutti i mezzi di trasporto sulla base di stime di percorrenza, su diversi tragitti di diverse lunghezze e a tutte le ore del giorno. Tra i risultati più interessanti si citano: l'automobile di proprietà vince su ogni percorso, a qualsiasi ora del giorno e per tutti i tipi di tragitti contro ogni mezzo di trasporto alternativo considerato; il car sharing \textit{Enjoy} ha perso il 35\% delle volte contro la bicicletta sulle tratte brevi dai 2 ai 5 km di lunghezza, percentuale che aumenta al 45\% se considerati solamente gli orari di picco del traffico delle 8:00 e 18:00; molto minore è la percentuale di sconfitte del car sharing da parte dei mezzi pubblici, di solo il 9\% delle volte.

\subsection{Quali insegnamenti si possono trarre dall'esperienza effettuata?}
Questa esperienza è stata molto educativa e significativa nel percorso di formazione professionale. Tra le principali competenze acquisite è sicuramente da citare quella legata all'ingegneria del software. Questo progetto infatti ha richiesto la progettazione e la suddivisione logica di diversi componenti software atti a raccogliere i dati dai siti web. Inizialmente è stato difficile, perché è stata la prima volta che si è passati dalla teoria alla pratica in proprio, senza aiuto di professori o altri compagni di corso. Un'altra competenza acquisita vi è quella della ricerca di articoli accademici e sviluppo di capacità critica. Personalmente infatti è stato molto difficile scontrarsi con i propri preconcetti e le proprie idee, che spesso sono state contraddette dalla realtà fatta di studi scientifici e argomentazioni logiche.

\subsection{Quali i problemi incontrati? Quali risolti e quali no? Perché?}
Sono stati incontrati numerosi problemi nello \textit{scraping}. Il primo sito web usato per raccogliere dati sui mezzi pubblici è stato \textit{Moovit}, che si è rivelato molto complesso e che ha richiesto un mese tra lo studio e la programmazione del modulo addetto alle richieste. Difatti, nella prima settimana di raccolta dati, ben l'80\% delle richieste di tragitto ha avuto risposta nulla, ovvero con stima di percorrenza pari a zero, difatti è stato subito scartato. Successivamente si è provveduto a reperire un altro servizio per i mezzi pubblici e sono state trovate le API messe a disposizione da \textit{Here}. Nonostante questo si sono riscontrati problemi anche qui, per colpa di un errore nella documentazione riguardo i parametri delle richieste da effettuare per ottenere i \textit{token}, ovvero delle parole chiave da inserire per avere accesso al servizio. Si è proceduto per tentativi e dopo qualche giorno si è riusciti a ottenere l'accesso. In generale, il grande problema di questo studio è stato quello del reperire servizi di navigazione basati su dati in tempo reale per avere delle stime di percorrenza molto precise, e solamente nel caso dei mezzi pubblici e del percorso in bici non si è riusciti nell'intento, accontentandosi solamente di stime basate su dati statici, ovvero sempre uguali indipendentemente dalle condizioni di traffico, dall'orario e dal giorno. Nel complesso però sono risultati sufficienti e in linea con le risorse adottate da altri studi.

%\section{Bibliografia}
\nocite{isfortaudimob}
\nocite{ellison2011travel}
\nocite{faghih2017hail}
\nocite{pagani2017knowledge}
\nocite{chien2003dynamic}

\bibliographystyle{plain}
\bibliography{Biblio}

\end{document}