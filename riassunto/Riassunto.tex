\documentclass{article}

\title{{\Large\textbf{CONFRONTO TRA AUTOMOBILE E MEZZI DI TRASPORTO ALTERNATIVI NEL COMUNE DI MILANO BASATO SU STIME DI TEMPI DI PERCORRENZA}}}
\author{\LARGE{Mauro Mastrapasqua - 892629}}
\date{\Large{27/01/2021}}


\begin{document}
	\pagenumbering{gobble}
	\maketitle
	\vspace{1cm}
	
\section{Ente presso cui è stato svolto il lavoro di stage}

\large{
Questo stage è stato un tirocinio interno svolto presso l'Università degli Studi di Milano, dipartimento di informatica, con relatore il professor Andrea Trentini e correlatore Dario Malchiodi. Vista la situazione di emergenza è stato svolto completamente nella propria personale residenza e con riunioni a distanza
}

\section{Contesto iniziale}

\large{
Nella sezione "proposte tesi" del sito web di Andrea Trentini veniva proposta una tesi dal seguente titolo: "Confronto fra mezzi pubblici e mezzi privati sulla base dei dati Google e ATM". Sono molti gli studi basati su dati relativi a un singolo mezzo prelevati tramite scraping di siti web a scopo di analisi generiche, soprattutto nel contesto mobilità condivisa, ma sono pochi invece quelli che hanno messo a confronto uno a uno diversi mezzi di trasporto, basandosi su questi dati, per capirne le dinamiche, come le performance nella breve e lunga distanza, o come l'accessibilità e il costo, o per sapere i margini di vittoria di uno o dell'altro mezzo.
}

\section{Obiettivi del lavoro}

\large{

}












	
\end{document}