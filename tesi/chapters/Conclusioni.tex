\section{Riflessione sul risultato}

Il risultato ottenuto dal confronto tra bicicletta e servizio di car sharing Enjoy sulla base dell'orario risulta molto interessante sotto diversi aspetti. Il primo è sicuramente la somiglianza del grafico stesso, \ref{image:12}, con quello delle prestazioni del car sharing nel grafico \ref{image:4}, che cattura la grande influenza della congestione stradale sull'efficienza in termini di tempo dell'uso dell'auto, sia essa di proprietà o condivisa. L'influenza del traffico sul servizio di car sharing preso in considerazione ha fatto sì che nelle ore intorno alle 8:00 e ancora intorno alle 18:00, lo stesso percorso è stato coperto in bicicletta nello stesso tempo se non più velocemente il 45\% delle volte, quasi 1 volta su 2. A giudicare da questo risultato, seppur prodotto sinteticamente, sembra che il servizio di car sharing riesca nell'intento di risolvere i primi due problemi principali dell'automobile, citati precedentemente, ovvero l'inquinamento, usando flotte di veicoli in linea con l'ultimo standard di emissioni euro e talvolta impiegando anche veicoli totalmente elettrici, e il costo, abbattuto dalla "condivisione" del mezzo, e quindi riducendo il pagamento al solo tempo di utilizzo. Ovviamente, il terzo problema riguardo l'efficienza rimane, risultando anche più pesante visto il tempo medio di 6 minuti per raggiungere la prima auto libera più vicina.

All'atto pratico però non risulta fattibile sostituire il servizio di car sharing o dei mezzi pubblici per fare così tanti chilometri pedalando, dato che si tratta di un'attività fisica medio-intensa e che può risultare stancante se fatta per più di 20 minuti. Tuttavia, per effettuare questo studio non si è preso in considerazione uno dei mezzi alternativi in gran diffusione nelle più grandi città del mondo, diretto concorrente della bicicletta e delle sue varianti a pedalata assistita: il monopattino elettrico.
Il monopattino elettrico infatti ha una velocità massima di 25 km/h, contro i 15 km/h di una pedalata normale in bicicletta, e può essere comprato per un costo che va dai 500 euro in sù oppure noleggiato (in mobilità condivisa) da uno dei tanti provider come Helbiz per €0.25 al minuto. Risulta molto più pratico da distribuire nella città e difatti sono molto diffusi, con un tempo medio per trovarne uno inferiore a quello del car sharing. Visti i suoi pregi e la sua velocità più alta rispetto alla bici, non sarebbe improbabile che tale mezzo riesca a battere il car sharing negli orari di punta più del 50\% delle volte e aumentando la percentuale anche nelle ore con meno traffico. Va inoltre fatta una nota riguardo l'economicità in caso di utilizzo di tale mezzo in mobilità condivisa: nel car sharing, si è presa in considerazione la lista delle auto libere solo ed esclusivamente di un servizio, Enjoy, di fatto aumentando il tempo di ricerca di un auto in mobilità condivisa in generale, per via di una quota fissa da pagare annualmente a ciascuno dei provider che si intende utilizzare, il che sposta la scelta del servizio al momento dell'iscrizione che per la maggior parte dei casi ricade su un unico provider. Al contrario, per i servizi sharing del monopattino non è richiesta nessuna documentazione ne quota fissa da pagare annualmente, ma si paga solo l'utilizzo.

\section{Possibili estensioni}