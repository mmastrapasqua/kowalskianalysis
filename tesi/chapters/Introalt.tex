L'automobile è il mezzo più diffuso al mondo, più della bicicletta. L'Italia è, da diversi anni, il secondo Paese in Europa col maggior numero di automobili per abitanti, con circa 640 auto ogni 1000 abitanti. Di queste, meno dello 0.4\% sono elettriche. Se consideriamo inoltre che al primo posto c'è il Lussemburgo, con una popolazione di 620,000 abitanti, l'Italia può essere considerata la prima in Europa.

Questo mezzo di trasporto ha diversi problemi, primo tra tutti è sicuramente l'inquinamento. I dati del 2018 dell'Automobile Club d'Italia (ACI) riportano come il 63\% delle auto in circolazione sia di categoria Euro4 o minore. Per affrontare questo problema, diverse città hanno messo in atto politiche per limitarne l'utilizzo in determinate aree, come l'introduzione dell'area C e dell'area B a Milano. Inoltre sono stati introdotti ecoincentivi per l'acquisto di auto elettriche che offrono uno sconto sull'acquisto e una tariffa agevolata sulle colonne di ricarica.

Secondo problema dell'auto è il costo. Una stima svolta da SosTariffe nel 2019 mostra come mantenere un'auto in Italia costi circa €1,620 all'anno, compresi di assicurazione, benzina e bollo auto, esclusi il prezzo del mezzo stesso, che si aggira intorno ai €10,000 per un'utilitaria a chilometro zero, e il cambio delle gomme.

Ultimo ma non ultimo dei problemi è l'efficienza in termini di tempo all'interno delle grandi città. 