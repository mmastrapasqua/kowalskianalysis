L'idea alla base di questo studio è, dunque, quella di confrontare diversi mezzi di trasporto su percorsi comuni, ovvero con punto di partenza, di destinazione, di data e orario di partenza uguali, salvando il tempo impiegato da ciascun mezzo per compiere il tragitto e usare successivamente tali valori per effettuare delle analisi, con l'obbiettivo finale di dichiarare un vincitore e, specialmente, capire di quanto abbia vinto. Infatti, non ci si aspetta un risultato diverso da quello che vede l'automobile come vincitore, ma si vuole analizzare se questo vantaggio sia relativamente grande dal punto di vista di una persona e se esistano delle variazioni lungo l'arco della giornata di questo vantaggio, visti i problemi relativi al traffico e di congestione stradale di cui le grandi città soffrono. Avendo a disposizione i dati relativi al percorso di confronto e al tempo impiegato dai vari mezzi, si possono effettuare ulteriori analisi, come le prestazioni di ogni mezzo in termini di velocità media, la variazione di ora in ora di quest'ultima o per esempio l'influenza che ha la lunghezza della tratta sulla velocità media.

Purtroppo però, un dataset contenente i tempi di percorrenza di tragitti comuni effettuati da diversi utenti su diversi mezzi di trasporto non esiste, molto probabilmente perchè nessuno l'ha mai veramente effettuato un confronto del genere. Gli unici che possiedono un dataset del genere di tratte realmente effettuate ma con un singolo mezzo sono le aziende private che offrono il servizio di navigazione, come per esempio Google Maps, Waze, Moovit, TomTom. Tali aziende infatti offrono applicazioni per poter guidare l'utente da un punto A ad un punto B con un determinato mezzo, per esempio mezzi pubblici, in tempo reale e col tracking GPS, e dato che il servizio nella maggior parte delle volte è offerto gratis, il pagamento per il servizio è effettuato coi dati del viaggio effettuato, e non hanno alcun incentivo nel pubblicare, anche in sola parte, i dati che hanno raccolto, nonostante siano i loro utenti i veri proprietari dei dati generati.

\section{Open Data}

La pubblicazione totale o parziale di questi dati è una pratica ancora poco diffusa che rientra nella filosofia dell'\emph{Open Data}. Secondo questa filosofia, simile a quella dell'\emph{open-source}, i dati dovrebbero essere liberamente accessibili da chiunque, senza nessun tipo di restrizioni derivanti da \emph{copyright}, patentini o altri meccanismi di controllo che riguarda la loro ridistribuzione. Uno dei campi che maggiormente beneficerebbe di questa filosofia sarebbe proprio la ricerca scientifica, che grazie all'accesso facilitato a questi dati riuscirebbe ad accelerare i tempi delle scoperte, e di conseguenza perfino le aziende stesse che possiedono questi dati ne gioverebbero. Non a caso sono già molti gli studi effettuati basati su dati acquisiti tramite scraping di siti web che sono riusciti a dare un contributo sia in ambito di ricerca e sia all'azienda stessa. Di questi vi è una sottocategoria dedicata ai trasporti. In questa categoria, un tema di spicco tra tutti è l'analisi dei dati prodotti dai servizi di mobilità condivisa. In uno studio effettuato a Milano nel 2014\cite{croci2014} sul bike sharing di ATM chiamato BikeMi, si è scoperto che nelle vicinanze di stazioni ferroviarie e università in zone di circolazione ristretta, il servizio veniva maggiormente usato rispetto ad altri posti. Un altro studio del 2017 effettuato sempre a Milano \cite{pagani2017} propone un sistema di nome Knowledge Discovery System basato su dati di 3 servizi di car sharing, Car2Go, Enjoy e Twist, in grado di offrire stime in tempo reale del percorso e di rilevare le zone in cui la richiesta del servizio è alta e le auto a disposizione scarseggiano. In questo studio inoltre viene evidenziato che durante il giorno le auto sono maggiormente concrentrate nella zona centrale di Milano mentre durante la sera si riversano nelle zone suburbane. In entrambi i due studi presi come esempio sono emerse delle informazioni che risulterebbero molto preziose per le aziende fornitrici dei servizi e che, di conseguenza, se venissero prese in considerazione, aumenterebbe lo sfruttamento del servizio da parte degli utenti.

Nonostance ciò, pochissime se non nessuna delle aziende che possiede dati generati dagli utenti ha mai reso disponibile apertamente quello che ha raccolto. Alcune di esse, come TomTom, si limitano a pubblicare regolarmente delle infografiche statistiche basate sui dati in loro possesso. Altre ancora si limitano a usare tali dati per autopromuoversi online e quindi farsi pubblicità. Queste pratiche però risultano problematiche sotto diversi aspetti, primo tra tutti è l'impossibilità di verificare le informazioni trasmesse, dato che la sorgente è nascosta, risultando quindi poco credibili considerando l'interesse primario di un'azienda in ambito economico. Un secondo problema, dal punto di vista dell'azienda, è quello di dover schierare, e quindi pagare, analisti di dati per produrre tali statistiche che però non hanno alcuna valenza scientifica, quando potrebbero solamente incaricarsi di rendere anonimi dei dati e pubblicarli, per ricevere gratuitamente studi fatti su tali dati da studenti e ricercatori di tutte le parti del mondo e che hanno una valenza scientifica.

\section{Prototipi}

Avendo l'accesso allo storico dei viaggi effettuati dagli utenti di più servizi, ognuno legato a un mezzo di trasporto, come Moovit per i mezzi pubblici e Waze per l'automobile, si potrebbero cercare delle tratte simili nei dataset di ognuno di essi, ovvero percorsi di viaggiatori che inconsapevolmente hanno condiviso punto di partenza e punto di arrivo, per confrontare i tempi di percorrenza. Tenendo conto che sarebbe poco efficiace usare un operatore di uguaglianza su delle coordinate geografiche prodotte da viaggiatori, dato che risulterebbero poche quelle che effettivamente coincidono, il che vorrebbe dire persone diverse che hanno fatto lo stesso identico viaggio, il modo migliore per catturare dei tragitti comuni sarebbe quello di considerare tali coordinate di partenza e di arrivo a gruppi, ovvero circoscritti entro un certo raggio e quindi considerabili come lo stesso viaggio. Per catturare questo concetto di gruppi si potrebbe utilizzare uno degli algoritmi di clusterizzazione, che sono stati creati apposta per questo tipo di problemi. Un possibile approccio infatti sarebbe quello di fondere i dataset dei diversi servizi in un unico dataset e clusterizzare i punti di partenza e di destinazione più frequenti usando l'algoritmo di clusterizzazione DBSCAN\cite{ester1996} o una delle sue varianti ottimizzate appositamente per dati geospaziali\cite{zhou2003}\cite{borah2004}, per mettere a confronto i mezzi di trasporto usando i viaggi appartenenti allo stesso cluster sia di partenza che di destinazione. Un approccio del genere però risulterebbe problematico sotto alcuni aspetti, primo tra tutti l'esecuzione dell'algoritmo stesso, infatti il DBSCAN ha una complessità temporale di $O(n^{2})$ e una complessità spaziale di $O(n)$ nel peggiore dei casi, complessità spaziale che arriva a $O(n^2)$ usando una matrice $n \times n$ per evitare delle ricomputazioni già effettuate. Per esempio, usando un campione di 100,000 viaggi, ognuno di essi contenente 2 variabili di tipo float32, e quindi di 4 Byte ciascuno, per le sole coordinate di partenza, eseguire tale algoritmo in tempi fattibili, e quindi utilizzando una tabella per le ricomputazioni, richiederebbe 











Lo stato attuale della tecnologia permette ai servizi di navigazione di avere una stima del tempo di percorrenza molto precisa e realistica, grazie a dati in tempo reale su traffico, incidenti, deviazioni di percorso e grazie allo storico dei tragitti percorsi dalle loro migliaia di utenti. Sfortunatamente, tali servizi non rendono pubblico tale storico, ma si limitano a pubblicare qualche analisi fatta su di esso. Inoltre, molti dei servizi sono specializzati nel fornire soluzioni per un solo mezzo.

\cite{croci2014}

\cite{rotaris2010}

\cite{rotaris2019}

\cite{meinardi2008}

\todo{atrent: da qualche parte ci sta una parentesi sui sistemi di navigazione crowd (es. waze) con cronistoria delle acquisizioni}

L'idea alla base di questo studio è stata quella di creare uno storico delle tratte tramite un'utenza fittizia, simulata scegliendo a random dei punti di partenza e destinazione, inoltrando tali punti come oggetto di richiesta per ogni servizio di navigazione corrispondente a un mezzo di trasporto diverso, al fine di confrontare a posteriori le performance di ogni mezzo e analizzarne le eventuali variazioni lungo l'arco della giornata.\todo{atrent: spiegare meglio, dichiarare cosa vuoi fare, quali parametri vorrai misurare e che tipo di paragoni vorrai fare}
