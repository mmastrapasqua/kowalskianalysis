L'idea alla base di questo studio è, dunque, quella di confrontare diversi mezzi di trasporto su percorsi comuni, ovvero con punto di partenza, di destinazione, di data e orario di partenza uguali, salvando il tempo impiegato da ciascun mezzo per compiere il tragitto e usare successivamente tali valori per effettuare delle analisi, con l'obbiettivo finale di analizzare le loro performance e compararle. Non ci si aspetta un risultato diverso da quello che vede l'automobile prima in classifica, ma si vuole analizzare se questo vantaggio sia relativamente grande dal punto di vista di una persona e se esistano delle variazioni lungo l'arco della giornata di questo vantaggio, visti i problemi relativi al traffico e di congestione stradale di cui le grandi città soffrono. Un'idea del possibile posizionamento in classifica la danno i dati ISFORT del 2019 basati su interviste a campione, riportati nella tabella \ref{table:9}, che mostrano le velocità medie percepite dagli utenti basate sui loro personali viaggi\cite{isfortaudimob}.

\begin{table}[H]
	\centering
	\begin{tabular}{ | l  r  r r | }
		\hline
		& \textbf{auto} & \textbf{bicicletta} & \textbf{mezzi pubblici} \\
		\textbf{V.media (km/h)} & 22 & 15 & 14 \\
		\hline
	\end{tabular}
	\caption{Dati ISFORT 2019, osservatorio "Audimob"}
	\label{table:9}
\end{table}

Avendo a disposizione i dati relativi al percorso di confronto e al tempo impiegato dai vari mezzi, sarebbe possibile ricreare la tabella per ogni fascia oraria, avendo l'opportunità di effettuare ulteriori analisi, come la variazione della velocità media di ora in ora per ogni mezzo o per esempio l'influenza che ha la lunghezza della tratta sulla velocità media. Purtroppo però, un dataset contenente i tempi di percorrenza di tragitti comuni effettuati da diversi utenti su diversi mezzi di trasporto non esiste, molto probabilmente perchè nessuno l'ha mai veramente effettuato un confronto del genere. Gli unici che possiedono dataset simili di tratte realmente effettuate ma con un singolo mezzo sono le aziende private che offrono il servizio di navigazione, come per esempio Google Maps, Waze, Moovit, TomTom. Tali aziende infatti offrono applicazioni per poter guidare l'utente da un punto A ad un punto B con un determinato mezzo, per esempio mezzi pubblici, in tempo reale e col tracking GPS, e dato che il servizio nella maggior parte delle volte è offerto gratis, il pagamento per tale servizio è effettuato coi dati del viaggio. Le aziende che offrono il servizio non hanno alcun incentivo nel pubblicare, anche in sola parte, i dati che hanno raccolto, nonostante siano i loro utenti i veri proprietari dei dati generati.

\section{Open Data}

La pubblicazione totale o parziale di questi dati è una pratica ancora poco diffusa che rientra nella filosofia dell'\emph{Open Data}. Secondo questa filosofia, simile a quella dell'\emph{open-source}, i dati dovrebbero essere liberamente accessibili da chiunque, senza nessun tipo di restrizioni derivanti da copyright, patentini o altri meccanismi di controllo che ostacolano la loro ridistribuzione. Uno dei campi che maggiormente beneficerebbe di questa filosofia sarebbe proprio la ricerca scientifica, che grazie all'accesso facilitato a questi dati riuscirebbe ad accelerare i tempi delle scoperte, e di conseguenza perfino le aziende stesse che possiedono questi dati ne gioverebbero. Non a caso sono già molti gli studi effettuati, basati su dati acquisiti tramite scraping di siti web, che sono riusciti a dare un contributo sia in ambito di ricerca e sia all'azienda stessa. Di questi vi è una sottocategoria dedicata ai trasporti. In questa categoria, un tema di spicco tra tutti è l'analisi dei dati prodotti dai servizi di mobilità condivisa. In uno studio effettuato a Milano nel 2014\cite{croci2014} sul bike sharing dell'azienda ATM chiamato BikeMi, si è osservato che nelle vicinanze di stazioni ferroviarie, università e zone a traffico limitato (ZTL), il servizio veniva maggiormente usato rispetto ad altri posti. Un altro studio del 2017 effettuato sempre a Milano \cite{pagani2017} propone un sistema di nome Knowledge Discovery System basato su dati di 3 servizi di car sharing, Car2Go, Enjoy e Twist, in grado di offrire stime in tempo reale dei tempi di percorrenza e di rilevare le zone in cui la richiesta del servizio è alta e le auto a disposizione scarseggiano. In questo studio inoltre viene evidenziato che durante il giorno le auto sono maggiormente concrentrate nella zona centrale di Milano mentre durante la sera si riversano nelle zone suburbane. In entrambi i due studi presi come esempio sono emerse delle informazioni che risulterebbero molto preziose per le aziende fornitrici dei servizi e che, di conseguenza, se venissero prese in considerazione, aumenterebbe lo sfruttamento del servizio da parte degli utenti, una soluzione win-win.

Nonostance ciò, pochissime se non nessuna delle aziende che possiede dati generati dagli utenti ha mai reso disponibile apertamente quello che ha raccolto. Alcune di esse, come TomTom, si limitano a pubblicare regolarmente delle infografiche e dei dati statistici basati sui dati in loro possesso. Altre ancora si limitano a usare tali dati per autopromuoversi online e quindi farsi pubblicità. Queste pratiche però risultano problematiche sotto diversi aspetti, primo tra tutti è l'impossibilità di verificare le informazioni trasmesse, dato che la sorgente è nascosta, risultando quindi poco credibili considerando l'interesse primario di un'azienda in ambito economico, che potrebbe fare di tali dati la sua pubblicità. Un secondo problema, dal punto di vista dell'azienda, è quello di dover schierare, e quindi pagare, analisti di dati per produrre tali statistiche che però non hanno alcuna valenza scientifica, quando potrebbero solamente incaricarsi di rendere anonimi dei dati e pubblicarli, per ricevere gratuitamente studi fatti su tali dati da studenti e ricercatori di tutte le parti del mondo e che hanno una valenza scientifica.

\section{Prototipo del progetto}

Avendo l'accesso allo storico dei viaggi effettuati dagli utenti di più servizi, ognuno legato a un mezzo di trasporto, come Moovit per i mezzi pubblici e Waze per l'automobile, si potrebbero cercare delle tratte simili nei dataset di ognuno di essi, ovvero percorsi di viaggiatori che inconsapevolmente hanno condiviso punto di partenza e punto di arrivo, per confrontare i tempi di percorrenza. Tenendo conto che sarebbe poco efficiace usare un operatore di uguaglianza su delle coordinate geografiche prodotte da viaggiatori, dato che risulterebbero poche quelle che effettivamente coincidono, il che vorrebbe dire persone diverse che hanno fatto lo stesso identico viaggio, il modo migliore per catturare dei tragitti comuni sarebbe quello di considerare tali coordinate di partenza e di arrivo a gruppi, ovvero circoscritti entro un certo raggio e quindi considerabili come lo stesso viaggio. Per catturare questo concetto di gruppi si potrebbe utilizzare uno degli algoritmi di clusterizzazione che sono stati pensati apposta per questo tipo di problemi. Un possibile approccio infatti sarebbe quello di fondere i dataset dei diversi servizi in un unico dataset e clusterizzare i punti di partenza e di destinazione più frequenti usando l'algoritmo di clusterizzazione DBSCAN\cite{ester1996} o una delle sue varianti ottimizzate appositamente per dati geospaziali\cite{zhou2003}\cite{borah2004}, che fanno della distanza tra i punti il fulcro dell'algoritmo, per mettere a confronto i mezzi di trasporto usando i viaggi appartenenti allo stesso cluster sia di partenza che di destinazione. Un approccio del genere però risulterebbe problematico sotto alcuni aspetti, primo tra tutti l'esecuzione dell'algoritmo stesso, infatti il DBSCAN ha una complessità temporale di $O(n^{2})$ e una complessità spaziale di $O(n)$ nel peggiore dei casi, complessità spaziale che arriva a $O(n^2)$ se si usa una matrice $n \times n$ per evitare delle ricomputazioni già effettuate. Per esempio, usando un campione di 100000 viaggi, ognuno di essi contenente 2 variabili di tipo float32, e quindi di 4 Byte ciascuno, per le sole coordinate di partenza, eseguire tale algoritmo in tempi fattibili, e quindi utilizzando una tabella per le ricomputazioni, servirebbero $100000^{2} \times (2 \times 4$B$) = 80 $GB di memoria centrale, il che richiederebbe il noleggio di un server per tale calcolo. Un altro problema legato a questo modo di procedere, indipendentemente dall'algoritmo usato per raggruppare le tratte, è quello di trovare dei viaggi percorsi nella stessa data e allo stesso orario, che ristringerebbe il campione dei viaggi drasticamente. Difatti avrebbe poco senso confrontare viaggi eseguiti in diversi giorni e/o a orari diversi per via del traffico che caratterizza una città, nello specifico quella di Milano. Secondo diversi servizi di monitoraggio del traffico infatti, un viaggio in auto in una settimana lavorativa il lunedì mattina alle 9:00 è molto sfavorito per via del picco di congestione stradale che caratterizza quel giorno, di conseguenza, confrontarlo con un qualsiasi altro viaggio che non abbia sofferto di questo rallentamento sarebbe poco significativo. Quello che si potrebbe fare avendo a disposizione un dataset di viaggi effettuati e che risulterebbe interessante sarebbe l'analisi delle performance di ogni mezzo preso singolarmente, misurandone diversi parametri, come velocità media, e rapportare quest'ultime analisi alle condizioni storiche del traffico per misurare l'impatto di queste condizioni sulle performance.

Non avendo nemmeno quest'ultima possibilità di avere accesso a dataset di viaggi compiuti con diversi mezzi e visti i problemi che avrebbe introdotto tale approccio, si è scelto di creare un dataset di gare tra mezzi attraverso una simulazione.

\section{La soluzione: simulazione}

Quado si effettua una richiesta a un servizio come Google Maps per un tragitto, quello che fa Google è calcolare, sulla base di dati in tempo reale del traffico, degli incidenti, degli orari di servizio e dei dati GPS, una stima del tempo di percorrenza per diversi mezzi di trasporto, come automobile, mezzi pubblici, tragitto a piedi e bicicletta (solo in presenza di piste ciclabili), per compiere tale tragitto. Sebbene siano solo stime e non cronometraggi realmente effettuati, esse risultano estremamente precise o quanto di più simile alla realtà. Lo stato attuale della tecnologia e la loro vasta e affermata presenza nella vita quotidiana delle persone permette a tali servizi di avere informazioni sempre più precise, utilizzando tecniche come quella del crowdsourcing di dati dagli smartphone degli utilizzatori e sfruttando lo storico dei viaggi effettuati. Dunque, si potrebbero usare questi dati non troppo lontani dalla realtà e costantemente aggiornati per confrontare i vari mezzi di trasporto su percorsi comuni, generati a caso. Non avendo altra scelta, quest'ultimo metodo è stato scelto definitivamente come metodo per questo studio.

\subsection{Crowdsourced data: il caso Waze}

Per fare un esempio, Waze, per fornire una stima del tempo di arrivo (in inglese nota come ETA: Estimated Time of Arrival), per prima cosa guarda i dati storici della velocità media di ogni segmento di strada compreso nell'itinerario calcolato. Questo particolare dato viene aggiornato ogni 30 minuti usando i dati prodotti dallo smartphone dei loro utenti, e nel caso non sia presente nessun dato la ricerca procede a ritroso fino alle ultime 8 settimane. In secondo luogo usa i dati in tempo reale che ha a disposizione degli utenti connessi e dei dati del traffico. Infine, tramite filtri per eliminare gli outlier, ovvero gli utenti estremamente veloci o lenti, ed equazioni matematiche danno una stima del tempo di percorrenza. Inoltre, tale stima viene aggiornata in tempo reale man mano che si procede nel viaggio, sempre tramite i dati collezionati da altri utenti. Quello che infatti ci tiene a sottolineare Waze e che la rende competitiva è la stima dei tempi di percorrenza basata interamente sulla loro community di utenti, dove ogni singola persona che usa la loro app contribuisce al bene di tutti migliorando le stime successive\cite{wazeblog}.






















