L'idea alla base di questo studio è, dunque, quella di confrontare diversi mezzi di trasporto su percorsi comuni, ovvero con punto di partenza, di destinazione, di data e orario di partenza uguali, salvando il tempo impiegato da ciascun mezzo per compiere il tragitto e usare successivamente tali valori per effettuare delle analisi, con l'obbiettivo finale di dichiarare un vincitore e, specialmente, capire di quanto abbia vinto. Infatti, non ci si aspetta un risultato diverso da quello che vede l'automobile come vincitore, ma si vuole analizzare se questo vantaggio sia relativamente grande dal punto di vista di una persona e se esistano delle variazioni lungo l'arco della giornata di questo vantaggio, visti i problemi relativi al traffico e di congestione stradale di cui le grandi città soffrono. Avendo a disposizione i dati relativi al percorso di confronto e al tempo impiegato dai vari mezzi, si possono effettuare ulteriori analisi, come le prestazioni di ogni mezzo in termini di velocità media, la variazione di ora in ora di quest'ultima o per esempio l'influenza che ha la lunghezza della tratta sulla velocità media.

Purtroppo però, un dataset contenente i tempi di percorrenza di tragitti comuni effettuati da diversi utenti su diversi mezzi di trasporto non esiste, molto probabilmente perchè nessuno l'ha mai veramente effettuato un confronto del genere. Gli unici che potrebbero possedere un dataset del genere di tratte realmente effettuate ma con un singolo mezzo sono le aziende private che offrono il servizio di navigazione, come per esempio Google Maps, Waze, Moovit, TomTom. Tali aziende infatti offrono applicazioni per poter guidare l'utente da un punto A ad un punto B con un determinato mezzo, per esempio mezzi pubblici, in tempo reale e col tracking GPS, e dato che il servizio nella maggior parte delle volte è offerto gratis, il pagamento per il servizio è effettuato coi dati del viaggio effettuato, e non hanno alcun incentivo nel pubblicare, anche in sola parte, i dati che hanno raccolto, nonostante siano i loro utenti i veri proprietari dei dati generati.


















Lo stato attuale della tecnologia permette ai servizi di navigazione di avere una stima del tempo di percorrenza molto precisa e realistica, grazie a dati in tempo reale su traffico, incidenti, deviazioni di percorso e grazie allo storico dei tragitti percorsi dalle loro migliaia di utenti. Sfortunatamente, tali servizi non rendono pubblico tale storico, ma si limitano a pubblicare qualche analisi fatta su di esso. Inoltre, molti dei servizi sono specializzati nel fornire soluzioni per un solo mezzo.

\cite{croci2014}

\cite{rotaris2010}

\cite{rotaris2019}

\cite{meinardi2008}

\todo{atrent: da qualche parte ci sta una parentesi sui sistemi di navigazione crowd (es. waze) con cronistoria delle acquisizioni}

L'idea alla base di questo studio è stata quella di creare uno storico delle tratte tramite un'utenza fittizia, simulata scegliendo a random dei punti di partenza e destinazione, inoltrando tali punti come oggetto di richiesta per ogni servizio di navigazione corrispondente a un mezzo di trasporto diverso, al fine di confrontare a posteriori le performance di ogni mezzo e analizzarne le eventuali variazioni lungo l'arco della giornata.\todo{atrent: spiegare meglio, dichiarare cosa vuoi fare, quali parametri vorrai misurare e che tipo di paragoni vorrai fare}
