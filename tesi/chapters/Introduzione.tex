L'\textbf{automobile} è il mezzo più diffuso al mondo, più della bicicletta. L'Italia è, da diversi anni, il secondo Paese in Europa col maggior numero di auto per abitanti, con circa 640 auto ogni 1000 abitanti\cite{eurostatcars}. Di queste, lo 0.4\% sono elettriche\cite{anfiastudiestatistiche}. Se consideriamo inoltre che al primo posto c'è il Lussemburgo, con una popolazione di 620000 abitanti, l'Italia può essere considerata la prima in Europa. 

\section{Problemi dell'automobile}

Questo mezzo di trasporto, sfortunatamente, ha diversi problemi.

Primo tra tutti è sicuramente l'\textbf{inquinamento}, che viene affrontato in tutte le fasi della vita di un'automobile, dagli standard dell'Unione Europea sulle emissioni a cui si devono attenere i produttori di auto per poter vendere sul territorio, dalle aziende di idrocarburi, che si impegnano a introdurre una percentuale di biocarburante nelle miscele, dalle politiche territoriali, come l'introduzione di aree a divieto di transito per veicoli inquinanti di una certa categoria, fino allo smaltimento del veicolo stesso, sempre regolato a livello legislativo. Nonostante queste numerose restrizioni e tentativi di contenere il problema, i risultati stentano ad arrivare. I dati del 2018 dell'Automobile Club d'Italia (ACI) riportano come il 63\% delle auto in circolazione sia di categoria Euro IV o minore\cite{anfiastudiestatistiche}, dove Euro IV è uno standard sulle emissioni introdotto nel 2006 e superato nel 2008 dall'Euro V, ben 12 anni fa\cite{euroivstandard}. Nel periodo 2008-2017 compresi, l'Italia ha superato in modo continuato i limiti giornalieri e annuali di livello di polveri sottili PM10, e per questo è stata multata dalla Corte di Giustizia dell'Unione Europea a pagare una cospicua multa\cite{eunewssanzioneitalia}. Se visto in prospettiva, l'inquinamento prodotto dalle auto non è una percentuale rilevante dell'inquinamento dell'aria, difatti è emerso in numerosissimi studi effettuati in diverse città del mondo, tra cui Milano e Brescia\cite{collivignarelli2020}\cite{camaletti2020}, durante periodo locale di lockdown per l'emergenza da COVID19, che anche a fronte di un traffico pressochè nullo, nell'aria si sono registrati gli stessi livelli di sostanze chimiche dannose nell'aria registrati nei periodi precedenti in condizioni di traffico abituale, con lievi abbassamenti riguardanti le sostanze riconducibili direttamente alla combustione dei motori termici. Questi studi infatti suggeriscono che la maggior parte dell'inquinamento dell'aria è legata a fattori ambientali. Una conclusione simile è stata ottenuta analizzando i dati dell'inquinamento dell'aria nelle aree dove è stata introdotta una restrizione sulla circolazione di alcuni veicoli, come l'area C nel comune di Milano (ex Ecopass), che concretamente ha dato scarsi risultati\cite{trentini2014}. Nonstante questo però, non bisogna dimenticare che, stando alle conoscenze attuali, i gas di scarico derivanti dai motori diesel risultano sicuri cancerogeni per l'uomo (classificato come gruppo 1 nella tabella IARC), al pari del fumo di tabacco, con la sottile differenza che quest'ultimo è qualcosa che si può evitare\cite{iarctable}. L'inquinamento dunque rimane un problema, dal momento che è maggiormente presente nelle città e negli agglomerati urbani, dove è presente la maggior parte della popolazione e dove le automobili, a livello di consumo di combustibile, risultano meno efficienti per via delle basse velocità.

Secondo problema dell'auto è il \textbf{costo}. Una stima svolta da SosTariffe nel 2019 mostra come mantenere un'auto in Italia costi circa €1,620 all'anno, compresi di assicurazione, benzina e bollo auto, esclusi il prezzo del mezzo stesso, che si aggira intorno ai €11,000 per un'utilitaria a chilometro zero, il cambio delle gomme ed eventuali imprevisti\cite{sostariffe}. Seppur questo problema risulti molto relativo, dato che è il reddito e il patrimonio di una persona a incidere sulla percezione di questa spesa, il solo costo di mantenimento di un'auto risulta molto più dispendioso rispetto ai mezzi alternativi, come un abbonamento annuale per i mezzi pubblici, per treni, o come l'acquisto e il mantenimento di una bicicletta o di mezzi alternativi. Inoltre, nel caso di percorrenza di grandi distanze, l'alternativa più veloce è l'impiego di tratte autostradali che attualmente in Italia sono di proprietà dalla società Autostrade per l'Italia, di cui uso è concesso tramite pagamento del pedaggio che è direttamente proporzionale alla distanza percorsa e risultra tra i più alti in Europa.

Ultimo ma non ultimo dei problemi è l'\textbf{efficienza} dell'auto in termini di tempo all'interno delle grandi città. Nell'ultimo studio annuale di TomTom del 2019\cite{tomtomindexmilan}, Roma, Palermo, Napoli, Messina e Milano rientrano tra le prime 100 città su scala globale per livello di congestione stradale, con un traffico tale da far raddoppiare i tempi di percorrenza durante le ore di punta del mattino e del tardo pomeriggio. Questo risultato può essere visto parzialmente come una concausa di un altro problema relativo a un fenomeno sociale, che è l'utilizzo dell'automobile per lo spostamento del solo conducente, nonostante la capacità maggiore dell'auto, che congestiona le strade per un flusso di persone relativamente minore di persone rispetto a un bus. A documentarlo è l'Associazione Nazionale Comuni Italiani (ANCI), che ha stimato che, degli 1.8 milioni di veicoli che si spostano quotidianamente nelle città italiane per tragitti casa-lavoro o casa-studio, 1.2 milioni viaggiano con solo il conducente a bordo\cite{anciperrepubblica}, più del 66\%.

\section{Vantaggi dell'automobile}

A giustificare la grande diffusione di questo mezzo di trasporto tra la popolazione sono sicuramente i grandi vantaggi che l'uso o il possesso di un auto comportano. Prima di tutto, possedere un auto può essere considerato sinonimo di libertà di muoversi:
\begin{itemize}
	\item non si hanno limiti di orario di partenza, perchè l'auto è pronta per essere usata a qualsiasi ora. La decisione dell'orario di inizio viaggio è esclusivamente del conducente, al contrario di treni e mezzi pubblici dove l'orario di partenza è offerto in numero limitato e prestabilito, con scarse percorrenze negli orari notturni;
	\item non si hanno limiti di distanza, perchè in automobile si può percorrere qualsiasi distanza senza particolari sforzi fisici, al contrario della bicicletta;
	\item non si hanno limiti di destinazione, perchè essenzialmente con l'automobile si può raggiungere qualsiasi posto raggiungibile con i mezzi alternativi e non, e in media più velocemente di quest'ultimi;
	\item si possono trasportare merci molto pesanti. Uno tra gli impieghi principali dell'auto, dopo quello del raggiungimento del luogo di lavoro e/o di studio, è quello di fare la spesa al supermercato e quello legato al trasporto di merce acquistata parecchio ingombrante, difficile da trasportare in assenza di un veicolo e spesso delegata a chi fornisce servizi di spedizioni.
\end{itemize}
Altri vantaggi legati all'auto sono legati alla sua comodità e praticità. Per esempio, nel Comune di Milano, la maggioranza dei parcheggi sono riservati ai residenti e sono divisi a zone, garantendo la possibilità di trovare parcheggio nei pressi della propria abitazione, sebbene tale provvedimento sia stato pensato originariamente a ostacolare gli spostamenti in auto all'interno della città. Inoltre, in caso di pioggia, temporale e persino neve, viaggiare in auto risulta l'opzione più comoda e sicura, e talvolta l'unica funzionante.























