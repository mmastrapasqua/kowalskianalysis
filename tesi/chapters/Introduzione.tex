Quando ci si deve spostare all'interno di una città, spesso ci si affida a servizi di navigazione per ottenere informazioni sul percorso più veloce da intraprendere insieme a una stima del tempo di percorrenza.
Il mezzo di trasporto nella maggior parte dei casi viene scelto a priori dall'utente per diversi motivi, quali abitudine, differenza di prezzo o mancanza di interesse per le alternative, anche se tale mezzo dovesse risultare il più lento.

Sebbene sia ragionevole pensare che l'auto di proprietà sia più veloce dei mezzi pubblici e delle biciclette, questo potrebbe non essere vero all'interno di una città come Milano. Infatti, secondo l'ultimo studio annuale di TomTom del 2019\cite{tomtomindexmilan}, Milano rientra tra le prime 100 città su scala globale per livello di congestione stradale, con un traffico tale da far quasi raddoppiare i tempi di percorrenza in auto durante le ore di punta del mattino e del tardo pomeriggio. In aggiunta al fatto che alcuni mezzi non soffrono minimamente di questo problema, come le metropolitane, o ne soffrono parzialmente, come le biciclette, questo rende l'ipotesi di partenza più discutibile.\todo{atrent: descrivere come viene calcolato indice TomTom, potrebbe calcolare un indice "strano" (a quanto ricordo paragona i tempi di percorrenza tra orari di punta e non quindi non è una misura assoluta)}

Lo stato attuale della tecnologia permette ai servizi di navigazione di avere una stima del tempo di percorrenza molto precisa e realistica, grazie a dati in tempo reale su traffico, incidenti, deviazioni di percorso e grazie allo storico dei tragitti percorsi dalle loro migliaia di utenti. Sfortunatamente, tali servizi non rendono pubblico tale storico, ma si limitano a pubblicare qualche analisi fatta su di esso. Inoltre, molti dei servizi sono specializzati nel fornire soluzioni per un solo mezzo.

\cite{croci2014}

\cite{rotaris2010}

\cite{rotaris2019}

\cite{meinardi2008}

\todo{atrent: da qualche parte ci sta una parentesi sui sistemi di navigazione crowd (es. waze) con cronistoria delle acquisizioni}

L'idea alla base di questo studio è stata quella di creare uno storico delle tratte tramite un'utenza fittizia, simulata scegliendo a random dei punti di partenza e destinazione, inoltrando tali punti come oggetto di richiesta per ogni servizio di navigazione corrispondente a un mezzo di trasporto diverso, al fine di confrontare a posteriori le performance di ogni mezzo e analizzarne le eventuali variazioni lungo l'arco della giornata.\todo{atrent: spiegare meglio, dichiarare cosa vuoi fare, quali parametri vorrai misurare e che tipo di paragoni vorrai fare}