L'automobile è il mezzo più diffuso al mondo, più della bicicletta. L'Italia è, da diversi anni, il secondo Paese in Europa col maggior numero di auto per abitanti, con circa 640 auto ogni 1000 abitanti\cite{eurostatcars}. Di queste, lo 0.4\% sono elettriche\cite{anfiastudiestatistiche}. Se consideriamo inoltre che al primo posto c'è il Lussemburgo, con una popolazione di 620000 abitanti, l'Italia può essere considerata la prima in Europa.

Questo mezzo di trasporto ha diversi problemi, primo tra tutti è sicuramente l'inquinamento, che nonostante venga affrontato direttamente dai produttori di auto con l'introduzione di motori e scarichi più efficienti, rimane comunque un problema per la loro lenta diffusione. I dati del 2018 dell'Automobile Club d'Italia (ACI) riportano come il 63\% delle auto in circolazione sia di categoria Euro IV o minore\cite{anfiastudiestatistiche}, dove Euro IV è uno standard sulle emissioni introdotto nel 2006 e superato nel 2008 dall'Euro V, ben 12 anni fa\cite{euroivstandard}. Nel periodo 2008-2017 compresi, l'Italia ha superato in modo continuato i limiti giornalieri e annuali di livello di polveri sottili PM10, e per questo è stata multata dalla Corte di Giustizia dell'Unione Europea a pagare una cospicua multa\cite{eunewssanzioneitalia}.

Secondo problema dell'auto è il \emph{costo}. Una stima svolta da SosTariffe nel 2019 mostra come mantenere un'auto in Italia costi circa €1,620 all'anno, compresi di assicurazione, benzina e bollo auto, esclusi il prezzo del mezzo stesso, che si aggira intorno ai €11,000 per un'utilitaria a chilometro zero, e il cambio delle gomme\cite{sostariffe}.

Ultimo ma non ultimo dei problemi è l'\emph{efficienza} dell'auto in termini di tempo all'interno delle grandi città. Nell'ultimo studio annuale di TomTom del 2019\cite{tomtomindexmilan}, Roma, Palermo, Napoli, Messina e Milano rientrano tra le prime 100 città su scala globale per livello di congestione stradale, con un traffico tale da far quasi raddoppiare i tempi di percorrenza durante le ore di punta del mattino e del tardo pomeriggio.

Preso atto di questi problemi legati alle automobili, in particolare l'ultimo legato all'efficienza, e preso in considerazione che, secondo quanto stimato dall'Associazione Nazionale Comuni Italiani (ANCI), degli 1.8 milioni di veicoli che si spostano quotidianamente nelle città italiane per tragitti casa-lavoro o casa-studio, 1.2 milioni viaggiano con solo il conducente a bordo\cite{anciperrepubblica}, viene spontaneo chiedersi: conviene davvero usare l'automobile per spostarsi all'interno delle città?