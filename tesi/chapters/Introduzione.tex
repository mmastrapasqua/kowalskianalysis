La mobilità a corto raggio è una percentuale dominante in Italia. È stato stimato che il 75\% degli spostamenti sul territorio italiano è lungo meno di 10 km, e di questi circa la metà sono inferiori ai 2 km \cite{isfortaudimob}. Il mezzo principale scelto per spostarsi all'interno delle aree urbane è l'automobile privata, che conta circa l'82\% del totale degli spostamenti. Non stupisce che da diversi anni l'Italia è il secondo Paese in Europa col maggior numero di automobili per abitanti, con circa 640 auto ogni 1000 persone, dietro solamente al Lussemburgo \cite{eurostatcars}.

\section{Automobile in città}

Nello studio annuale di TomTom del 2019 Milano rientra tra le prime 110 città su scala globale per livello di congestione stradale, con un traffico tale da far aumentare di quasi il doppio il tempo impiegato a percorrere una tratta durante le ore di picco\footnote{\url{https://www.tomtom.com/en_gb/traffic-index/ranking/}. Accessed: 20/01/2021}. Molti sono stati i cambiamenti apportati a questa città per fronteggiare la domanda crescente di spostamenti e al contempo diminuire il traffico stradale, tra queste si citano la realizzazione della Linea M5, l'introduzione dell'area C e delle numerose zone a traffico limitato (ZTL), l'aumento di piste ciclabili, fino alla concessione ad aziende nel campo della mobilità condivisa.

A fronte di questi dati sul traffico e delle restrizioni introdotte viene spontaneo chiedersi quanto sia influenzato il tempo di percorrenza in automobile su strade cittadine e, soprattutto, se questa influenza sia tale da favorire altri mezzi di trasporto. Sebbene sia un dato di fatto che, su strada, un'automobile sia più veloce di un tram o di una bicicletta, dall'altra bisogna tener conto dei vantaggi di cui godono gli altri mezzi. Basti pensare alle numerose strade con corsia preferenziale o carreggiata a parte per bus e tram, e a mezzi che non soffrono minimamente dei problemi del traffico, come le metropolitane ATM e i passanti ferroviari Trenord.

L'obbiettivo dello studio è quello di analizzare a livello di tempistiche l'impiego dell'automobile privata a motore termico, che rappresenta il 99.5\% del parco auto\footnote{\label{footanfia}\url{https://www.anfia.it/it/automobile-in-cifre/}. Accessed: 20/01/2021}, per tratte brevi all'interno del Comune di Milano e confrontarlo con mezzi di trasporto alternativi, al fine di capire se l'influenza delle congestioni che caratterizzano la città è tale da favorire uno di questi. Le principali motivazioni risiedono nel costo di questo mezzo e nei problemi derivati dal suo largo impiego.

\section{Problemi}

%Nel periodo 2008-2017 compresi, l'Italia ha superato in modo continuato i limiti giornalieri e annuali di livello di polveri sottili PM10, e per questo è stata multata dalla Corte di Giustizia dell'Unione Europea a pagare una cospicua multa\cite{eunewssanzioneitalia}.

L'automobile a motore termico ha diversi problemi. Primo tra tutti è sicuramente l'\textbf{inquinamento}. I dati del 2018 dell'Automobile Club d'Italia (ACI) riportano come il 63\% delle auto in circolazione sia di categoria Euro IV o minore, uno standard sulle emissioni introdotto nel 2006 e superato nel 2008 dall'Euro V, ben 12 anni fa\textsuperscript{\ref{footanfia}}. Se visto in prospettiva, l'inquinamento prodotto dalle auto non è una percentuale rilevante dell'inquinamento dell'aria, difatti è emerso in numerosissimi studi effettuati in diverse città del mondo, tra cui Milano e Brescia \cite{collivignarelli2021analysis}\cite{cameletti2020effect}, che, durante il periodo locale di lockdown per l'emergenza da COVID-19,  quindi a fronte di un traffico pressoché nullo, nell'aria si sono registrati gli stessi livelli di sostanze chimiche nocive dei periodi precedenti in condizioni di traffico abituale, con lievi abbassamenti riguardanti le sostanze riconducibili direttamente alla combustione dei motori termici. Questi studi suggeriscono che la maggior parte dell'inquinamento dell'aria è legata a fattori ambientali. Una conclusione simile è stata ottenuta analizzando i dati dell'inquinamento all'interno dell'area C del Comune di Milano (ex Ecopass), un'area in cui sono state introdotte delle restrizioni sulla circolazione di alcuni veicoli, che concretamente ha dato scarsi risultati \cite{trentini2014lombardy}. Nonostante questo però non bisogna dimenticare che, per esempio, i gas di scarico derivanti dai motori diesel risultano sicuri cancerogeni per l'uomo, al pari del fumo di tabacco, con la sottile differenza che quest'ultimo è qualcosa che si può evitare\footnote{\url{https://monographs.iarc.who.int/}. Accessed: 20/01/2021}. L'inquinamento dunque rimane un problema in Italia, visto il parco auto ancora poco aggiornato e dal momento che la maggior parte del traffico è presente nelle città, dove le automobili, a livello di consumo di combustibile, risultano meno efficienti per via delle basse velocità e delle numerose accelerazioni.

Secondo problema dell'automobile, se comparato con gli altri mezzi di trasporto per uso cittadino, è il \textbf{costo}. Una stima svolta da SosTariffe\todo{atrent: refbib? MM: footnote, Sos è il riferimento italiano su stime dei costi, ma rilascia solo comunicati stampa} nel 2019 mostra come mantenere un'auto in Italia costi circa €1800 all'anno, compresi di assicurazione, benzina e bollo auto, senza contare il prezzo iniziale del mezzo, che si aggira intorno ai €11000 per un'utilitaria a chilometro zero, il cambio delle gomme ed eventuali imprevisti\footnote{\url{https://www.sostariffe.it/info/supporto-giornalistil}. Accessed: 20/01/21}. Seppur questo problema risulti molto relativo, dato che è il reddito e patrimonio di una persona a incidere sulla percezione di questa spesa, il solo costo di mantenimento di un'auto risulta molto più dispendioso rispetto alle sue alternative in città, come un abbonamento annuale per i mezzi pubblici o come l'acquisto e il mantenimento di una bicicletta o di mezzi alternativi.\todo{atrent: il costo assoluto per km percorso non necessariamente è un fattore fondamentale per la scelta, esistono anche altri fattori come la sicurezza fisica (incidente, furto del portafoglio sui mezzi, protezione da covid) trasporto di cose, indipendenza dagli orari o disponibilità sotto casa MM: sotto riporto questi vantaggi}

Ultimo ma non ultimo dei problemi è l'\textbf{efficienza}dell'automobile in termini di tempo, che viene ridotta per via delle congestioni stradali ricorrenti che caratterizzano le città. Tra i contro dello spostamento in auto, a livello collettivo, vi è sicuramente l'ingombro stradale rapportato al numero di persone in viaggio, ovvero l'utilizzo dell'automobile per lo spostamento del solo conducente, nonostante la capacità maggiore dell'auto, tradotto nella congestione delle strade per un flusso di persone molto inferiore rispetto a quello di un bus. A documentarlo è l'Associazione Nazionale Comuni Italiani (ANCI), stimando che, degli 1.8 milioni di veicoli che si spostano quotidianamente nelle città italiane per tragitti casa-lavoro o casa-studio, 1.2 milioni viaggiano con solo il conducente a bordo\footnote{\url{http://www.anci.it/mobilita-sostenibile-ricerca-anci-nelle-citta-serpentoni-di-auto-vuote-ma-cala-inquinamento/}. Accessed: 20/01/2021}.

\section{Vantaggi}

A giustificare la grande diffusione di questo mezzo di trasporto tra la popolazione sono sicuramente i grandi vantaggi che si acquisiscono dal possederne una. L'auto può essere considerata sinonimo di libertà di muoversi: non si hanno limiti di orario di partenza; la decisione dell'orario di inizio viaggio è esclusivamente del conducente, al contrario dei mezzi pubblici dove l'orario di partenza è offerto in numero limitato e prestabilito, con scarse percorrenze negli orari notturni; si possono trasportare merci molto pesanti, difatti uno tra gli impieghi principali dell'auto, dopo quello del raggiungimento del luogo di lavoro e/o di studio, è quello di fare la spesa al supermercato e/o di trasportare merce ingombrante acquistata, difficile da attuare in assenza di un veicolo. Inoltre, in caso di forte pioggia o maltempo, viaggiare in auto risulta l'opzione più comoda e sicura, talvolta l'unica funzionante. Al momento dunque la comodità e la praticità di un automobile sono senza pari.

\section{Confronto con mezzi alternativi}

È sul terzo problema che si concentra questo studio, che ha come obbiettivo capire quanto sia vantaggioso o svantaggioso in termini di tempi di percorrenza, a fronte di problemi quali traffico e congestioni stradali, usare un'automobile all'interno della città di Milano, dal punto di vista di un'utenza con sole esigenze di spostamento, senza vincoli legati a trasporto merci, intemperie o raggiungimento di luoghi al di fuori della città. Per capire quale sia il mezzo più vantaggioso la soluzione ideale sarebbe quella di organizzare diverse gare in diversi punti della città in cui delle persone, ognuna con un mezzo di trasporto diverso, vengono cronometrate su un percorso cittadino comune, ovvero partendo da uno stesso punto di partenza e arrivando a uno stesso punto di arrivo, più volte al giorno e per diversi percorsi, al fine di raccogliere un campione abbastanza significativo di dati sul quale effettuare i confronti. Purtroppo, non avendo a disposizione le risorse e il tempo per effettuare un confronto del genere, si è optato per una soluzione informatica.