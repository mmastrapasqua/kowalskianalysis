L'\textbf{automobile} è il mezzo più diffuso al mondo, più della bicicletta. L'Italia è, da diversi anni, il secondo Paese in Europa col maggior numero di auto per abitanti, con circa 640 auto ogni 1000 abitanti\cite{eurostatcars}. Di queste, lo 0.4\% sono elettriche\cite{anfiastudiestatistiche}. Se consideriamo inoltre che al primo posto c'è il Lussemburgo, con una popolazione di 620000 abitanti, l'Italia può essere considerata la prima in Europa. 

\section{I problemi dell'automobile}

Questo mezzo di trasporto, sfortunatamente, ha diversi problemi.

Primo tra tutti è sicuramente l'\textbf{inquinamento}, che viene affrontato in tutte le fasi della vita di un'automobile, dagli standard dell'Unione Europea sulle emissioni a cui si devono attenere i produttori di auto per poter vendere sul territorio, dalle aziende di idrocarburi, che si impegnano a introdurre una percentuale di biocarburante nelle miscele, dalle politiche territoriali, come l'introduzione di aree a divieto di transito per veicoli inquinanti di una certa categoria, fino allo smaltimento del veicolo stesso, sempre regolato a livello legislativo. Nonostante queste numerose restrizioni e tentativi di contenere il problema, i risultati stentano ad arrivare. I dati del 2018 dell'Automobile Club d'Italia (ACI) riportano come il 63\% delle auto in circolazione sia di categoria Euro IV o minore\cite{anfiastudiestatistiche}, dove Euro IV è uno standard sulle emissioni introdotto nel 2006 e superato nel 2008 dall'Euro V, ben 12 anni fa\cite{euroivstandard}.

\cite{trentini2014}

 Nel periodo 2008-2017 compresi, l'Italia ha superato in modo continuato i limiti giornalieri e annuali di livello di polveri sottili PM10, e per questo è stata multata dalla Corte di Giustizia dell'Unione Europea a pagare una cospicua multa\cite{eunewssanzioneitalia}. Se visto in prospettiva, l'inquinamento prodotto dalle auto non è una percentuale rilevante dell'inquinamento dell'aria, difatti è emerso in numerosissimi studi effettuati in diverse città del mondo, tra cui Milano e Brescia\cite{collivignarelli2020}\cite{camaletti2020}, durante periodo locale di lockdown per l'emergenza da COVID19, che anche a fronte di un traffico pressochè nullo, nell'aria si sono registrati gli stessi livelli di sostanze chimiche dannose nell'aria registrati nei periodi precedenti in condizioni di traffico abituale, con lievi abbassamenti riguardanti le sostanze riconducibili direttamente alla combustione dei motori termici. Tali studi infatti suggeriscono che tali valori siano legati più a fattori ambientali. Nonstante questo però, non bisogna dimenticare che, stando alle conoscenze attuali, i gas di scarico derivanti dai motori diesel risultano sicuri cancerogeni per l'uomo (classificato come gruppo 1 nella tabella IARC), al pari del fumo di tabacco, con la sottile differenza che quest'ultimo è qualcosa che si può evitare\cite{iarctable}.




Secondo problema dell'auto è il costo. Una stima svolta da SosTariffe nel 2019 mostra come mantenere un'auto in Italia costi circa €1,620 all'anno, compresi di assicurazione, benzina e bollo auto, esclusi il prezzo del mezzo stesso, che si aggira intorno ai €11,000 per un'utilitaria a chilometro zero, e il cambio delle gomme\cite{sostariffe}.

Ultimo ma non ultimo dei problemi è l'\textbf{efficienza} dell'auto in termini di tempo all'interno delle grandi città. Nell'ultimo studio annuale di TomTom del 2019\cite{tomtomindexmilan}, Roma, Palermo, Napoli, Messina e Milano rientrano tra le prime 100 città su scala globale per livello di congestione stradale, con un traffico tale da far quasi raddoppiare i tempi di percorrenza durante le ore di punta del mattino e del tardo pomeriggio.

Preso atto di questi problemi legati alle automobili, , viene spontaneo chiedersi: conviene davvero usare l'automobile per spostarsi all'interno delle città?
\todo{atrent: io sarò sempre convinto che l'auto (e la moto ancora di più) CONVIENE rispetto a qualunque cosa, anche se costa di più, non è solo un tema economico! non puoi ignorare il tema!}