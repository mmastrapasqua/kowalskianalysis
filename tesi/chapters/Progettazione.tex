\section{Progettazione dello studio}

Per poter confrontare ...

\section{Mezzi di trasporto e relativi servizi}

Sono stati presi in considerazione i principali mezzi di trasporto con cui è possibile spostarsi all'interno del Comune di Milano, e per ognuno di essi è stato selezionato un servizio di navigazione in grado di offrire stime del tempo di percorrenza in tempo reale per tale mezzo. Nello specifico:

\begin{itemize}
	\item mezzi pubblici ATM e passanti ferroviari;
	\begin{itemize}
		\item HERE WeGo REST API\cite{herewegoapi}
	\end{itemize}
	\item automobile di proprietà;
	\begin{itemize}
		\item Waze API\cite{wazeapi}
	\end{itemize}
	\item bicicletta di proprietà;
	\begin{itemize}
		\item OpenStreetMap\cite{openstreetmap}
	\end{itemize}
	\item a piedi;
	\begin{itemize}
		\item OpenStreetMap\cite{openstreetmap}
	\end{itemize}
	\item car sharing Enjoy.
	\begin{itemize}
		\item Enjoy Map\cite{enjoycarsharing}
	\end{itemize}
\end{itemize}

I servizi di navigazione HERE, Waze e OpenStreetMap offrono delle API per effettuare delle richieste nella seguente forma:
\begin{lstlisting}[language=Go]
func stimaTragitto(a, b coordinate) durata
\end{lstlisting}
dove richiedono in input le coordinate di partenza e di destinazione e restituiscono in output una stima di percorrenza del tragitto in formato JSON.
\todo{MM: inserire accenno su come ho trovato l'auto più vicina}
Il servizio di Enjoy offre solamente una lista aggiornata delle auto libere presenti nella città. Le stime di percorrenza basate su tale mezzo sono state calcolate utilizzando il servizio di OpenStreetMap e di Waze. Nello specifico, date le coordinate di partenza del tragitto, viene richiesto a OpenStreetMap di calcolare il tempo di percorrenza a piedi verso la macchina più vicina dal punto di partenza. Una volta trovata l'auto più vicina, viene richiesto a Waze di calcolare il restante tempo di percorrenza a partire dalle coordinate della macchina fino al punto di destinazione. Nei risultati viene salvata la somma dei due tempi di percorrenza.


\section{Scelta tra percorsi random e prefissati}

Si è scelto di usare diverse tratte su cui basare il confronto allo scopo di ricoprire a livello topografico una buona parte del Comune di Milano e di diversificare i confronti in base alle caratteristiche delle tratte, quali lunghezza, coordinate geografiche di partenza e destinazione, per simulare al meglio la posizione di un possibile utente nella mappa.

Per rendere automatiche le richieste da inoltrare ai servizi di navigazione è stata creata una sorgente dal quale pescare le tratte. Sono stati presi in considerazione tre principali approcci per creare tale sorgente: hard-coded random; hard-coded di tratte realmente percorse; generazione a random just-in-time. I primi due approcci sono risultati fin da subito problematici.

In primo luogo, la scelta della destinazione avrebbe introdotto bias cognitivi riguardo il possibile utente del percorso, portando ad analizzare la mobilità solamente dal punto di vista di una determinata categoria, per esempio: selezionando tratte con delle università come punto di arrivo porterebbe ad analizzare la mobilità solamente dal punto di vista degli studenti e dipendenti presso quelle determinate strutture. Altri problemi simili sarebbero sorti nello scegliere la partenza, la lunghezza, la distanza dal centro città e numerosità delle tratte.

Seconda problematica molto più rilevante è che, involontariamente, si sarebbero introdotti dei percorsi che avrebbero favorito un mezzo piuttosto che un altro. Nella città di Milano infatti si contanto numerosi tratti stradali di questo genere: strade con corsia preferenziale per mezzi pubblici e taxi; strade a singola corsia e con numerosi semafori (quindi più soggetta a incolonnamenti); tangenziali; tratte coperte da passanti ferroviari (più veloci delle metropolitane e con meno fermate). Tale scelta avrebbe portato ad analizzare dati non rappresentativi della città, con conseguenti risultati in un certo senso "falsati".

Usando l'approccio della generazione a random non solo si sarebbero evitati tali problemi, ma i problemi stessi si sarebbero trasformati in analisi da poter eseguire a posteriori, per esempio selezionando da tutte le tratte generate a random quelle che hanno portato nei pressi di un'università, o tratte che in linea d'aria hanno coperto particolari strade favorevoli a determinati mezzi di trasporto. Visti i vantaggi e la flessibilità dell'approccio, si è optato per quest'ultimo.

\section{Generatore random dei percorsi}

Prima ancora di scrivere una funzione per generare tratte a random è stata scelta l'area geografica all'interno della quale generarle. Siccome il servizio di car sharing Enjoy non permette di usare la propria flotta al di fuori del Comune di Milano, tale area è stata selezionata come terreno per i confronti, rappresentata sotto forma di rettangolo per questioni di semplicità nella programmazione. Nello specifico, sono state scelte le coordinate (45.450562\textdegree, 9.158959\textdegree) e (45.482032\textdegree, 9.206763\textdegree) rispettivamente come vertice in basso a sinistra e in alto a destra del rettangolo rappresentativo dell'area selezionata. In questo modo la generazione di punti a random è stata ridotta alla generazione di coordinate maggiori o uguali della prima e minori o uguali della seconda.

Una volta scritta la funzione per la generazione a random delle tratte, è stato introdotto un constraint per simulare dei percorsi scomodi da fare a piedi, ovvero per generare tratte che supererebbero i 20 minuti di camminata per essere coperte, e al tempo stesso che giustificherebbero l'utilizzo della macchina. Si è scelto di usare 2 km in linea d'aria dal punto di partenza a quello di destinazione come misura minima della lunghezza di una tratta.

Un secondo constraint è stato introdotto per avere una maggiore eterogeneità delle tratte generate a livello geografico. Molte delle linee principali dei mezzi pubblici infatti attraversano il centro storico di Milano, inoltre l'area del centro storico rappresenta più della metà dell'area selezionata dal constraint precedente. Siccome è stato scelto di ricoprire a livello topografico tutta l'area di Milano, si è programmato il generatore in modo da creare in maniera equidistribuita le seguenti tipologie di tratte: da dentro il centro storico a fuori; da dentro a dentro; da fuori a fuori; da fuori a dentro. Il rettangolo rappresentativo del centro storico è stato disegnato secondo i seguenti vertici: (45.450562\textdegree, 9.158959\textdegree), (45.482032\textdegree, 9.206763\textdegree).

\section{Timing delle richieste}

Le richieste sono state programmate per essere effettuate dalle 7:00 alle 23:59 di ogni giorno. La scelta di tale intervallo è stata vincolata dall'orario di servizio dei mezzi pubblici ATM, difatti in tale orario viene garantito il pieno regime del servizio, mentre vengono offerti collegamenti e passaggi saltuari al di fuori di esso.
Per rispettare i limiti giornalieri delle varie API si è scelto di effettuare 1 richiesta al minuto, dove ogni richiesta rappresenta un tragitto generato a random mandato simultaneamente a ogni servizio di navigazione per ottenere una stima da ognuno di essi, per un totale di circa 1,000 richieste al giorno. Oltre ai valori delle stime di percorrenza per ogni mezzo, insieme alla richiesta sono stati salvati altri dati come il numero di macchine libere enjoy al momento della richiesta, la distanza aerea della tratta e quella calcolata a piedi. 

\section{Pseudocodice del programma}

\begin{lstlisting}[language=Go]
var a, b, c coordinate

for {
	a, b = creaTragittoRandom()
	c = enjoy.trovaAutoPiuVicina(a)
	
	risultato := []stime{
		here.stimaTragitto(a, b),
		waze.stimaTragitto(a, b),
		osm.stimaTragitto(a, b, "foot"),
		osm.stimaTragitto(a, b, "bike"),
		osm.stimaTragitto(a, c, "foot") +
			waze.stimaTragitto(c, b)
	}
	
	save(risultato)
}
\end{lstlisting}