Il primo problema citato riguardo l'inquinamento dell'automobile è un problema che è stato largamente affrontato e viene affrontato tutt'ora da diversi enti pubblici e privati di tutto il mondo, tramite studi mirati e introduzione di nuove norme per affrontare direttamente il problema, come per esempio gli standard Euro per i nuovi veicoli prodotti, l'introduzione di percentuali di biodiesel nei carburanti, fino ad arrivare a politiche territoriali come il divieto di circolazione di alcune categorie di veicoli all'interno di particolari aree, come ad esempio l'introduzione dell'area C e B nella città di Milano.\todo{atrent: che non hanno alcun effetto sull'aria, ho pubblicato in proposito, si prega di citare}

Il secondo problema, riguardo il costo del mezzo nella sua interezza, non è un problema in se, ma il risultato di un paragone fatto con altri mezzi di trasporto, come mezzi pubblici, bicicletta e treni, che risultano nettamente più economici. Inoltre è un problema relativo, che varia in base al reddito di una persona.\todo{atrent: ribadisco, NON è un tema solo economico, si tratta anche di comodità e praticità}

È sul terzo problema che si concentra questo studio, che ha come obbiettivo capire quanto sia vantaggioso o svantaggioso in termini di tempi di percorrenza usare un'automobile all'interno della città di Milano a fronte di problemi come traffico e congestioni stradali.
Sebbene sia un dato di fatto che un'automobile viaggi più veloce di un bus o di una bicicletta, questo potrebbe non realizzarsi all'interno di una grande città. Inoltre, tali mezzi alternativi contano numerosi vantaggi in termini di percorso, basti pensare alle numerose strade con corsia preferenziale o carreggiata a parte per bus e tram, alle lunghe piste ciclabili che ultimamente sono state create per favorire la mobilità alternativa, fino a mezzi che non soffrono minimamente dei problemi del traffico, come le metropolitane e i passanti ferroviari Trenord. Soprattutto a fronte di questi vantaggi che l'ipotesi di partenza, ovvero che in auto si viaggia più veloce degli altri mezzi alternativi all'interno di una città, risulta più discutibile.\todo{atrent: questo lo devi ancora dimostrare! va messo tutto in ipotetico}

Per capire quale sia il mezzo più vantaggioso, la soluzione ideale sarebbe quella di organizzare diverse gare in diversi punti della città, in cui delle persone, ognuno con un mezzo di trasporto diverso, partono da un comune punto di partenza e raggiungono un comune punto di destinazione, cronometrando il tempo impiegato da ciascuno. Purtroppo, non avendo a disposizione le risorse e il tempo (e, date le restrizioni per la pandemia di COVID19, nemmeno il permesso) per effettuare un confronto del genere, si è optato per una soluzione informatica.

\todo{atrent: se proprio vuoi fare dei confronti in termini di vantaggi e svantaggi devi definirli tutti (velocità, capacità di carico, sicurezza, protezione dalle intemperie, ecc. se no non è scientifico)}