Il primo problema citato riguardo l'inquinamento dell'automobile è un problema che è stato largamente affrontato e viene affrontato tutt'ora da diversi enti pubblici e privati di tutto il mondo, tramite studi mirati e introduzione di nuove norme per affrontare direttamente il problema, come per esempio gli standard Euro per i nuovi veicoli prodotti, fino ad arrivare a politiche territoriali come il divieto di circolazione di alcune categorie di veicoli all'interno di particolari aree, come ad esempio l'introduzione dell'area C e B nella città di Milano.

Il secondo problema, riguardo il costo del mezzo nella sua interezza, non è un problema in se, ma il risultato di un paragone fatto con altri mezzi di trasporto, come mezzi pubblici, bicicletta e treni, che risultano nettamente più economici. Inoltre è un problema relativo, che varia in base al reddito di una persona.

È sul terzo problema 

Sebbene sia ragionevole pensare che l'auto di proprietà sia più veloce dei mezzi pubblici e delle biciclette, questo potrebbe non essere vero all'interno di una città come Milano. Infatti, secondo l'ultimo studio annuale di TomTom del 2019\cite{tomtomindexmilan}, Milano rientra tra le prime 100 città su scala globale per livello di congestione stradale, con un traffico tale da far quasi raddoppiare i tempi di percorrenza in auto durante le ore di punta del mattino e del tardo pomeriggio. In aggiunta al fatto che alcuni mezzi non soffrono minimamente di questo problema, come le metropolitane, o ne soffrono parzialmente, come le biciclette, questo rende l'ipotesi di partenza più discutibile.