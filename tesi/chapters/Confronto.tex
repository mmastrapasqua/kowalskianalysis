Possedere e usare un automobile porta sicuramente numerosi ed enormi vantaggi. I problemi riguardanti l'inquinamento sono per la maggior parte risolti a livello legislativo e meccanico dai produttori dell'automobile, per cui non interessano direttamente l'utente. Il secondo problema, come già citato, è molto relativo e personale, guidato spesso da concrete esigenze come per esempio un luogo di lavoro distante, spostamenti frequente di merci pesanti, e per tanto non può considerarsi un problema quanto, il prezzo da pagare per ottenere certi benefici. E' il terzo problema, riguardante l'efficienza in termini di tempo, che è un problema serio per chi percorre abitualmente strade cittadine e su cui l'utente non ha nessun controllo.



È sul terzo problema che si concentra questo studio, che ha come obbiettivo capire quanto sia vantaggioso o svantaggioso in termini di tempi di percorrenza usare un'automobile all'interno della città di Milano a fronte di problemi come traffico e congestioni stradali.
Sebbene sia un dato di fatto che un'automobile viaggi più veloce di un bus o di una bicicletta, questo potrebbe non realizzarsi all'interno di una grande città. Inoltre, tali mezzi alternativi contano numerosi vantaggi in termini di percorso, basti pensare alle numerose strade con corsia preferenziale o carreggiata a parte per bus e tram, alle lunghe piste ciclabili che ultimamente sono state create per favorire la mobilità alternativa, fino a mezzi che non soffrono minimamente dei problemi del traffico, come le metropolitane e i passanti ferroviari Trenord. Soprattutto a fronte di questi vantaggi che l'ipotesi di partenza, ovvero che in auto si viaggia più veloce degli altri mezzi alternativi all'interno di una città, risulta più discutibile.\todo{atrent: questo lo devi ancora dimostrare! va messo tutto in ipotetico}

Per capire quale sia il mezzo più vantaggioso, la soluzione ideale sarebbe quella di organizzare diverse gare in diversi punti della città, in cui delle persone, ognuno con un mezzo di trasporto diverso, partono da un comune punto di partenza e raggiungono un comune punto di destinazione, cronometrando il tempo impiegato da ciascuno. Purtroppo, non avendo a disposizione le risorse e il tempo (e, date le restrizioni per la pandemia di COVID19, nemmeno il permesso) per effettuare un confronto del genere, si è optato per una soluzione informatica.

\todo{atrent: se proprio vuoi fare dei confronti in termini di vantaggi e svantaggi devi definirli tutti (velocità, capacità di carico, sicurezza, protezione dalle intemperie, ecc. se no non è scientifico)}